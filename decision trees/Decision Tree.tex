\documentclass{mlnotes}

\lecture{7}
\topic{Decision Trees}
\author{Aru Gyani}
\subtitle{Based on Prof.\ Rishabh Iyer's slides.}

% --- Links --- %
\def\seeingtheory{https://seeing-theory.brown.edu/basic-probability/index.html}
\def\khanstats{https://www.khanacademy.org/math/ap-statistics/probability-ap/stats-conditional-probability/a/check-independence-conditional-probability}

\begin{document}
\maketitle
\tableofcontents
\clearpage

\part{Probability, Random Variables, and Entropy}
\section{Probability}
The following content is sourced from the following:
\begin{itemize}
  \item \link{\seeingtheory}{seeing-theory}
  \item \link{\khanstats}{Khan Academy}
  \item Professor Rishabh Iyer's class notes
\end{itemize}

\subsection{Discrete Probability}

\begin{definition}{Sample space}{sample_space}
  \define{Sample space} specifies the set of possible outcomes.\\[12pt]
  For example, \(\Omega=\{H,T\}\) would be the set of possible outcomes of a
  coin flip. Simply put, we are saying a coin flip could be either
  \emph{heads} or \emph{tails}.
\end{definition}

\begin{definition}{Probability}{probability}
  \define{Probability} is simply how likely something is to happen\\[12pt]
  For each element \(\omega\) inside of our sample space \(\Omega\), there is a
  number \(p(\omega)\ \epsilon\ [0,1]\) called a probability. This represents
  how likely the event \(\omega\) is to happen.\\[12pt]
  The total probability of all possible outcomes (each outcome being an element
  \(\omega\)) within the sample space \(\Omega\) adds up to \(1\). This means
  that one of the outcomes in the sample space is certain to occur.
  
  \[
    \sum_{\omega \in \Omega} p(\omega) = 1
  \]

  For exaxmple, a biased coin might have \(p(H) = .6\) and \(p(T) = .4\).
  \\[12pt]
  \small{Note: \([0,1]\)} is a range, not a set. This number is always between 0
  and 1, where 0 indicates impossibility and 1 indicates certainty.
\end{definition}

\begin{definition}{Event}{event}
  An \define{event} is a subset of the sample space \(\Omega\), e.g.
  \begin{itemize}
    \item Let \(\Omega = \{1,2,3,4,5,6\}\) be the \(6\) possible outcomes of a
    dice roll.
    \item \(A = \{1,5,6\} \subseteq \Omega\) would be the event that the dice roll
    comes up as a one, five, or six.
  \end{itemize}

  The probability of an event is just the \underline{sum of all the outcomes
  that it contains}.
  \[
    p(A) = p(1) + p(5) + p(6)
  \]
  \small{Note: \(A \subseteq\ \Omega\) means that \(A\) is a subset of \(\Omega\). Below, is a brief recap of what a subset is.}
  \begin{definition}{Subset}{subset}
    A set \(A\) is a\ \define{subset} of another set \(B\) if all of
    elements of the set \(A\) are elements of the set \(B\). In other words, the
    set \(A\) is contained inside the set \(B\).
  \end{definition} 
\end{definition}

\subsection{Conditional Probability}
\subsubsection{Independence}
We say two events are independent if knowing one event occurred doesn't change
the probability of the other event.
For example, the probability that a fair coin shows ``heads'' after being
flipped is \(\frac{1}{2}\). What if we knew the day was Tuesday? Does this
change the probability of getting ``heads''? Of course not. The probability of
getting ``heads'', given that it's a Tuesday, is still \(\frac{1}{2}\). Thus,
the result of a coin flip and the day being Tuesday are independent events.
\link[blue]{\khanstats}{More reading}.

\begin{definition}{\define{Independence}}{independence}
  Two events \(A\) and \(B\) are independent if
  \[
    p(A \cap B) = p(A)P(B) 
  \]

  Informally, \(p(A \cap B)\) indicates the probability of \(A\) and \(B\) (aka
  probability of \(A\) intersection \(B\)). This means that it indicates the likelihood of both events happening.
\end{definition}

Let's suppose that we have a fair die: \(p(1) = \cdots = p(6) = \frac{1}{6}\) (the
probability of rolling any number on the dice is equal). If \(A = \{1,2,5\}\)
and \(B = \{3,4,6\}\) are \(A\) and \(B\) independent?

\section{Discrete Random Variables}
\section{Entropy}
\subsection{Conditional Entropy}
\subsection{Information Gain}

\part{Decision Trees}

\phantomsection{}
\addcontentsline{toc}{part}{Index}
\printindex[defn]
\end{document}